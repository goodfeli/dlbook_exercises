%!TEX root = all_exercises.tex

\beforeFirstExercises{ch3}
\section{Why Probability?}

add exercises here...

\section{Random Variables}

\begin{exercises}{ch3}

	\begin{exercise} 
		EXRCIS  QUESTION

		\begin{eanswer}EXRCIS  ANS\end{eanswer}
		\begin{esolution}
			EXRCIS SOLUTION.
		\end{esolution}
	\end{exercise}


\end{exercises}

\section{Probability Distributions}

\begin{exercises}{ch3}

	\begin{exercise} 
		Do the following vectors represent probability distributions?
		\begin{exparts*}
			\partsitem	$\left(\tfrac{1}{2}, \tfrac{1}{2}, \tfrac{1}{2} \right)^T$
			\partsitem	$\left(\tfrac{1}{4}, \tfrac{1}{4}, \tfrac{1}{4}, \tfrac{1}{4} \right)^T$
			\partsitem	$\left(0.3, 0.3, -0.1, 0.5  \right)^T$
		\end{exparts*}

		\begin{eanswer}\begin{ansparts*}
					\partsitem No; weights don't add to one.
					\partsitem Yes.
					\partsitem No; contains a negative number.
					\end{ansparts*}\end{eanswer}
	\end{exercise}


\end{exercises}






\setcounter{section}{13}
\section{Structured Probabilistic Models}

add exercises here...


\afterLastExercises{ch3}



\setcounter{section}{14}
\section{Probability and Information Theory Problems}

\begin{problems}{ch3}


	\begin{problem}		\label{problem:geometric_distr_biased_coin_until_heads}
		You have a biased coin which lands on \texttt{heads} with probability $p$,
		and consequently lands on \texttt{tails} with probability $(1-p)$.
		Suppose you want to flip the coin until you get \texttt{heads}.
		Define the random variable $N$ as the number of tosses required until the first \texttt{heads} outcome.
		What is the probability mass function $P_N(n)$ for success on the $n$\textsuperscript{th} toss?
		Confirm that the formula is a valid probability distribution by showing $\sum_{n=1}^\infty P_N(n) = 1$.

		\begin{hint}
			Find the probabilities for cases $n=1,2,3,\ldots$ and look for a pattern.
		\end{hint}

		\begin{solution}
			The biased coin flip is modelled by a random variable $Y$,
			and different coin flips correspond to random variables $Y_1$, $Y_2$, $Y_3$, \ldots which are independent copies of $Y$.
			The probability of getting \texttt{heads} on the first flip is $P_N(1)=\textrm{Pr}\!\left( \{ Y_1=\texttt{heads} \} \right)\! =p$.
			The probability of getting \texttt{heads} on the second flip corresponds
			to the event $\{Y_1=\texttt{tails}\} \ \texttt{AND} \ \{Y_2=\texttt{heads} \}$.
			We assumed the coin flips are independent so % the probability of this event is the product
			$P_N(2)=(1-p)p$.
			Similarly $P_N(3) = (1-p)^2p$.
			The general formula is $P_N(n) = (1-p)^{n-1}p$.
		\end{solution}
	\end{problem}


\end{problems}


