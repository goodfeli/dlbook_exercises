\documentclass{article}
\usepackage{bm}
\newif\ifMIT
%\MITtrue % uncomment to use MIT template
\MITfalse
\usepackage{amsmath}
\usepackage{amsfonts}
\usepackage{amsthm}
%%%%% NEW MATH DEFINITIONS %%%%%



\def\ceil#1{\lceil #1 \rceil}
\def\floor#1{\lfloor #1 \rfloor}
\def\1{\bm{1}}
\newcommand{\train}{\mathcal{D}}
\newcommand{\valid}{\mathcal{D_{\mathrm{valid}}}}
\newcommand{\test}{\mathcal{D_{\mathrm{test}}}}

\def\eps{{\epsilon}}


% Random variables
\def\reta{{\textnormal{$\eta$}}}
\def\ra{{\textnormal{a}}}
\def\rb{{\textnormal{b}}}
\def\rc{{\textnormal{c}}}
\def\rd{{\textnormal{d}}}
\def\re{{\textnormal{e}}}
\def\rf{{\textnormal{f}}}
\def\rg{{\textnormal{g}}}
\def\rh{{\textnormal{h}}}
\def\ri{{\textnormal{i}}}
\def\rj{{\textnormal{j}}}
\def\rk{{\textnormal{k}}}
\def\rl{{\textnormal{l}}}
%\def\rm{{\textnormal{m}}}
\def\rn{{\textnormal{n}}}
\def\ro{{\textnormal{o}}}
\def\rp{{\textnormal{p}}}
\def\rq{{\textnormal{q}}}
\def\rr{{\textnormal{r}}}
\def\rs{{\textnormal{s}}}
\def\rt{{\textnormal{t}}}
\def\ru{{\textnormal{u}}}
\def\rv{{\textnormal{v}}}
\def\rw{{\textnormal{w}}}
\def\rx{{\textnormal{x}}}
\def\ry{{\textnormal{y}}}
\def\rz{{\textnormal{z}}}

% Random vectors
\def\rvepsilon{{\mathbf{\epsilon}}}
\def\rvtheta{{\mathbf{\theta}}}
\def\rva{{\mathbf{a}}}
\def\rvb{{\mathbf{b}}}
\def\rvc{{\mathbf{c}}}
\def\rvd{{\mathbf{d}}}
\def\rve{{\mathbf{e}}}
\def\rvf{{\mathbf{f}}}
\def\rvg{{\mathbf{g}}}
\def\rvh{{\mathbf{h}}}
\def\rvu{{\mathbf{i}}}
\def\rvj{{\mathbf{j}}}
\def\rvk{{\mathbf{k}}}
\def\rvl{{\mathbf{l}}}
\def\rvm{{\mathbf{m}}}
\def\rvn{{\mathbf{n}}}
\def\rvo{{\mathbf{o}}}
\def\rvp{{\mathbf{p}}}
\def\rvq{{\mathbf{q}}}
\def\rvr{{\mathbf{r}}}
\def\rvs{{\mathbf{s}}}
\def\rvt{{\mathbf{t}}}
\def\rvu{{\mathbf{u}}}
\def\rvv{{\mathbf{v}}}
\def\rvw{{\mathbf{w}}}
\def\rvx{{\mathbf{x}}}
\def\rvy{{\mathbf{y}}}
\def\rvz{{\mathbf{z}}}

% Elements of random vectors
\def\erva{{\textnormal{a}}}
\def\ervb{{\textnormal{b}}}
\def\ervc{{\textnormal{c}}}
\def\ervd{{\textnormal{d}}}
\def\erve{{\textnormal{e}}}
\def\ervf{{\textnormal{f}}}
\def\ervg{{\textnormal{g}}}
\def\ervh{{\textnormal{h}}}
\def\ervi{{\textnormal{i}}}
\def\ervj{{\textnormal{j}}}
\def\ervk{{\textnormal{k}}}
\def\ervl{{\textnormal{l}}}
\def\ervm{{\textnormal{m}}}
\def\ervn{{\textnormal{n}}}
\def\ervo{{\textnormal{o}}}
\def\ervp{{\textnormal{p}}}
\def\ervq{{\textnormal{q}}}
\def\ervr{{\textnormal{r}}}
\def\ervs{{\textnormal{s}}}
\def\ervt{{\textnormal{t}}}
\def\ervu{{\textnormal{u}}}
\def\ervv{{\textnormal{v}}}
\def\ervw{{\textnormal{w}}}
\def\ervx{{\textnormal{x}}}
\def\ervy{{\textnormal{y}}}
\def\ervz{{\textnormal{z}}}

% Random matrices
\def\rmA{{\mathbf{A}}}
\def\rmB{{\mathbf{B}}}
\def\rmC{{\mathbf{C}}}
\def\rmD{{\mathbf{D}}}
\def\rmE{{\mathbf{E}}}
\def\rmF{{\mathbf{F}}}
\def\rmG{{\mathbf{G}}}
\def\rmH{{\mathbf{H}}}
\def\rmI{{\mathbf{I}}}
\def\rmJ{{\mathbf{J}}}
\def\rmK{{\mathbf{K}}}
\def\rmL{{\mathbf{L}}}
\def\rmM{{\mathbf{M}}}
\def\rmN{{\mathbf{N}}}
\def\rmO{{\mathbf{O}}}
\def\rmP{{\mathbf{P}}}
\def\rmQ{{\mathbf{Q}}}
\def\rmR{{\mathbf{R}}}
\def\rmS{{\mathbf{S}}}
\def\rmT{{\mathbf{T}}}
\def\rmU{{\mathbf{U}}}
\def\rmV{{\mathbf{V}}}
\def\rmW{{\mathbf{W}}}
\def\rmX{{\mathbf{X}}}
\def\rmY{{\mathbf{Y}}}
\def\rmZ{{\mathbf{Z}}}

% Elements of random matrices
\def\ermA{{\textnormal{A}}}
\def\ermB{{\textnormal{B}}}
\def\ermC{{\textnormal{C}}}
\def\ermD{{\textnormal{D}}}
\def\ermE{{\textnormal{E}}}
\def\ermF{{\textnormal{F}}}
\def\ermG{{\textnormal{G}}}
\def\ermH{{\textnormal{H}}}
\def\ermI{{\textnormal{I}}}
\def\ermJ{{\textnormal{J}}}
\def\ermK{{\textnormal{K}}}
\def\ermL{{\textnormal{L}}}
\def\ermM{{\textnormal{M}}}
\def\ermN{{\textnormal{N}}}
\def\ermO{{\textnormal{O}}}
\def\ermP{{\textnormal{P}}}
\def\ermQ{{\textnormal{Q}}}
\def\ermR{{\textnormal{R}}}
\def\ermS{{\textnormal{S}}}
\def\ermT{{\textnormal{T}}}
\def\ermU{{\textnormal{U}}}
\def\ermV{{\textnormal{V}}}
\def\ermW{{\textnormal{W}}}
\def\ermX{{\textnormal{X}}}
\def\ermY{{\textnormal{Y}}}
\def\ermZ{{\textnormal{Z}}}

% Vectors
\def\vzero{{\bm{0}}}
\def\vone{{\bm{1}}}
\def\va{{\bm{a}}}
\def\vb{{\bm{b}}}
\def\vc{{\bm{c}}}
\def\vd{{\bm{d}}}
\def\ve{{\bm{e}}}
\def\vf{{\bm{f}}}
\def\vg{{\bm{g}}}
\def\vh{{\bm{h}}}
\def\vi{{\bm{i}}}
\def\vj{{\bm{j}}}
\def\vk{{\bm{k}}}
\def\vl{{\bm{l}}}
\def\vm{{\bm{m}}}
\def\vn{{\bm{n}}}
\def\vo{{\bm{o}}}
\def\vp{{\bm{p}}}
\def\vq{{\bm{q}}}
\def\vr{{\bm{r}}}
\def\vs{{\bm{s}}}
\def\vt{{\bm{t}}}
\def\vu{{\bm{u}}}
\def\vv{{\bm{v}}}
\def\vw{{\bm{w}}}
\def\vx{{\bm{x}}}
\def\vy{{\bm{y}}}
\def\vz{{\bm{z}}}
\ifMIT
% MIT_TEMPLATE : \bm{\lambda} was generating missing fonts,
% probably because \newcommand{\bmmax}{1} removed them.
% I think \boldmath should be very similar.
\def\vlambda{{\boldmath{\lambda}}}
\def\vmu{{\boldmath{\mu}}}
\def\valpha{{\boldmath{\alpha}}}
\def\vbeta{{\boldmath{\beta}}}
\def\gamma{{\boldmath{\vgamma}}}
\def\vtheta{{\boldmath{\theta}}}
\def\vepsilon{{\boldmath{\epsilon}}}
\def\vomega{{\boldmath{\omega}}}
\def\vpsi{{\boldmath{\psi}}}
\def\vsigma{{\boldmath{\sigma}}}
\def\vSigma{{\boldmath{\Sigma}}}
\def\veta{{\boldmath{\eta}}}
\def\vrho{{\boldmath{\rho}}}
\def\vRho{{\boldmath{\Rho}}}
\def\vphi{{\boldmath{\phi}}}
\def\vPhi{{\boldmath{\Phi}}}
\def\vDelta{{\boldmath{\Delta}}}
\def\vdelta{{\boldmath{\delta}}}
\else
\def\vlambda{{\bm{\lambda}}}
\def\vmu{{\bm{\mu}}}
\def\valpha{{\bm{\alpha}}}
\def\vbeta{{\bm{\beta}}}
\def\vgamma{{\bm{\gamma}}}
\def\vtheta{{\bm{\theta}}}
\def\vepsilon{{\bm{\epsilon}}}
\def\vomega{{\bm{\omega}}}
\def\vpsi{{\bm{\psi}}}
\def\vphi{{\bm{\phi}}}
\def\vsigma{{\bm{\sigma}}}
\def\vSigma{{\bm{\Sigma}}}
\def\veta{{\bm{\eta}}}
\def\vrho{{\bm{\rho}}}
\def\vRho{{\bm{\Rho}}}
\def\vphi{{\bm{\phi}}}
\def\vPhi{{\bm{\Phi}}}
\def\vDelta{{\bm{\Delta}}}
\def\vdelta{{\bm{\delta}}}
\fi

%\def\KeyIdea{{\marginpar{KEY IDEA}}}
% YB: until we populate the book more thoroughly with KEY IDEA, probably best to leave this out
\def\KeyIdea{}

% Elements of vectors
\def\evalpha{{\alpha}}
\def\evbeta{{\beta}}
\def\evepsilon{{\epsilon}}
\def\evlambda{{\lambda}}
\def\evomega{{\omega}}
\def\evmu{{\mu}}
\def\evpsi{{\psi}}
\def\evsigma{{\sigma}}
\def\evtheta{{\theta}}
\def\eva{{a}}
\def\evb{{b}}
\def\evc{{c}}
\def\evd{{d}}
\def\eve{{e}}
\def\evf{{f}}
\def\evg{{g}}
\def\evh{{h}}
\def\evi{{i}}
\def\evj{{j}}
\def\evk{{k}}
\def\evl{{l}}
\def\evm{{m}}
\def\evn{{n}}
\def\evo{{o}}
\def\evp{{p}}
\def\evq{{q}}
\def\evr{{r}}
\def\evs{{s}}
\def\evt{{t}}
\def\evu{{u}}
\def\evv{{v}}
\def\evw{{w}}
\def\evx{{x}}
\def\evy{{y}}
\def\evz{{z}}

% Matrix
\def\mA{{\bm{A}}}
\def\mB{{\bm{B}}}
\def\mC{{\bm{C}}}
\def\mD{{\bm{D}}}
\def\mE{{\bm{E}}}
\def\mF{{\bm{F}}}
\def\mG{{\bm{G}}}
\def\mH{{\bm{H}}}
\def\mI{{\bm{I}}}
\def\mJ{{\bm{J}}}
\def\mK{{\bm{K}}}
\def\mL{{\bm{L}}}
\def\mM{{\bm{M}}}
\def\mN{{\bm{N}}}
\def\mO{{\bm{O}}}
\def\mP{{\bm{P}}}
\def\mQ{{\bm{Q}}}
\def\mR{{\bm{R}}}
\def\mS{{\bm{S}}}
\def\mT{{\bm{T}}}
\def\mU{{\bm{U}}}
\def\mV{{\bm{V}}}
\def\mW{{\bm{W}}}
\def\mX{{\bm{X}}}
\def\mY{{\bm{Y}}}
\def\mZ{{\bm{Z}}}
\ifMIT
\def\mLambda{{\boldmath{\Lambda}}}
\def\mBeta{{\boldmath{\beta}}}
\def\mPhi{{\boldmath{\Phi}}}
\def\mSigma{{\boldmath{\Sigma}}}
\else
\def\mBeta{{\bm{\beta}}}
\def\mPhi{{\bm{\Phi}}}
\def\mLambda{{\bm{\Lambda}}}
\def\mSigma{{\bm{\Sigma}}}
\fi

% Tensor
% Got this from here: http://tex.stackexchange.com/questions/77640/bold-italic-and-sans-serif-math-symbols
\DeclareMathAlphabet{\mathsfit}{\encodingdefault}{\sfdefault}{m}{sl}
\SetMathAlphabet{\mathsfit}{bold}{\encodingdefault}{\sfdefault}{bx}{n}
\newcommand{\tens}[1]{\bm{\mathsfit{#1}}}
\def\tA{{\tens{A}}}
\def\tB{{\tens{B}}}
\def\tC{{\tens{C}}}
\def\tD{{\tens{D}}}
\def\tE{{\tens{E}}}
\def\tF{{\tens{F}}}
\def\tG{{\tens{G}}}
\def\tH{{\tens{H}}}
\def\tI{{\tens{I}}}
\def\tJ{{\tens{J}}}
\def\tK{{\tens{K}}}
\def\tL{{\tens{L}}}
\def\tM{{\tens{M}}}
\def\tN{{\tens{N}}}
\def\tO{{\tens{O}}}
\def\tP{{\tens{P}}}
\def\tQ{{\tens{Q}}}
\def\tR{{\tens{R}}}
\def\tS{{\tens{S}}}
\def\tT{{\tens{T}}}
\def\tU{{\tens{U}}}
\def\tV{{\tens{V}}}
\def\tW{{\tens{W}}}
\def\tX{{\tens{X}}}
\def\tY{{\tens{Y}}}
\def\tZ{{\tens{Z}}}


% Graph
\def\gA{{\mathcal{A}}}
\def\gB{{\mathcal{B}}}
\def\gC{{\mathcal{C}}}
\def\gD{{\mathcal{D}}}
\def\gE{{\mathcal{E}}}
\def\gF{{\mathcal{F}}}
\def\gG{{\mathcal{G}}}
\def\gH{{\mathcal{H}}}
\def\gI{{\mathcal{I}}}
\def\gJ{{\mathcal{J}}}
\def\gK{{\mathcal{K}}}
\def\gL{{\mathcal{L}}}
\def\gM{{\mathcal{M}}}
\def\gN{{\mathcal{N}}}
\def\gO{{\mathcal{O}}}
\def\gP{{\mathcal{P}}}
\def\gQ{{\mathcal{Q}}}
\def\gR{{\mathcal{R}}}
\def\gS{{\mathcal{S}}}
\def\gT{{\mathcal{T}}}
\def\gU{{\mathcal{U}}}
\def\gV{{\mathcal{V}}}
\def\gW{{\mathcal{W}}}
\def\gX{{\mathcal{X}}}
\def\gY{{\mathcal{Y}}}
\def\gZ{{\mathcal{Z}}}

% Sets
\def\sA{{\mathbb{A}}}
\def\sB{{\mathbb{B}}}
\def\sC{{\mathbb{C}}}
\def\sD{{\mathbb{D}}}
% Don't use a set called E, because this would be the same as our symbol
% for expectation.
\def\sF{{\mathbb{F}}}
\def\sG{{\mathbb{G}}}
\def\sH{{\mathbb{H}}}
\def\sI{{\mathbb{I}}}
\def\sJ{{\mathbb{J}}}
\def\sK{{\mathbb{K}}}
\def\sL{{\mathbb{L}}}
\def\sM{{\mathbb{M}}}
\def\sN{{\mathbb{N}}}
\def\sO{{\mathbb{O}}}
\def\sP{{\mathbb{P}}}
\def\sQ{{\mathbb{Q}}}
\def\sR{{\mathbb{R}}}
\def\sS{{\mathbb{S}}}
\def\sT{{\mathbb{T}}}
\def\sU{{\mathbb{U}}}
\def\sV{{\mathbb{V}}}
\def\sW{{\mathbb{W}}}
\def\sX{{\mathbb{X}}}
\def\sY{{\mathbb{Y}}}
\def\sZ{{\mathbb{Z}}}

% Matrix elements
% NOTE: we changed these from lower case to upper case to avoid
% clashes between vectors and matrices with same letter name.
% For example, we don't want elements of the hidden layer vector
% h to have the same name as elements of the Hessian matrix H
\def\emLambda{{\Lambda}}
\def\emA{{A}}
\def\emB{{B}}
\def\emC{{C}}
\def\emD{{D}}
\def\emE{{E}}
\def\emF{{F}}
\def\emG{{G}}
\def\emH{{H}}
\def\emI{{I}}
\def\emJ{{J}}
\def\emK{{K}}
\def\emL{{L}}
\def\emM{{M}}
\def\emN{{N}}
\def\emO{{O}}
\def\emP{{P}}
\def\emQ{{Q}}
\def\emR{{R}}
\def\emS{{S}}
\def\emT{{T}}
\def\emU{{U}}
\def\emV{{V}}
\def\emW{{W}}
\def\emX{{X}}
\def\emY{{Y}}
\def\emZ{{Z}}
\def\emSigma{{\Sigma}}

% Tensor elements
% Same font as tensor, without \bm wrapper
\newcommand{\etens}[1]{\mathsfit{#1}}
\def\etLambda{{\etens{\Lambda}}}
\def\etA{{\etens{A}}}
\def\etB{{\etens{B}}}
\def\etC{{\etens{C}}}
\def\etD{{\etens{D}}}
\def\etE{{\etens{E}}}
\def\etF{{\etens{F}}}
\def\etG{{\etens{G}}}
\def\etH{{\etens{H}}}
\def\etI{{\etens{I}}}
\def\etJ{{\etens{J}}}
\def\etK{{\etens{K}}}
\def\etL{{\etens{L}}}
\def\etM{{\etens{M}}}
\def\etN{{\etens{N}}}
\def\etO{{\etens{O}}}
\def\etP{{\etens{P}}}
\def\etQ{{\etens{Q}}}
\def\etR{{\etens{R}}}
\def\etS{{\etens{S}}}
\def\etT{{\etens{T}}}
\def\etU{{\etens{U}}}
\def\etV{{\etens{V}}}
\def\etW{{\etens{W}}}
\def\etX{{\etens{X}}}
\def\etY{{\etens{Y}}}
\def\etZ{{\etens{Z}}}

% The true underlying data-generating distribution
\newcommand{\pdata}{p_{\rm{data}}}
% The empirical distribution defined by the training set
\newcommand{\ptrain}{\hat{p}_{\rm{data}}}
\def\empirical{\ptrain} % Accidental duplicate
\newcommand{\Ptrain}{\hat{P}_{\rm{data}}}
% The model distribution
\newcommand{\pmodel}{p_{\rm{model}}}
\newcommand{\Pmodel}{P_{\rm{model}}}
\newcommand{\ptildemodel}{\tilde{p}_{\rm{model}}}
% Stochastic autoencoder distributions
\newcommand{\pencode}{p_{\rm{encoder}}}
\newcommand{\pdecode}{p_{\rm{decoder}}}
\newcommand{\precons}{p_{\rm{reconstruct}}}

\newcommand{\laplace}{\mathrm{Laplace}} % Laplace distribution, see first use in prob.tex

\newcommand{\E}{\mathbb{E}}
\newcommand{\Ls}{\mathcal{L}}
\newcommand{\R}{\mathbb{R}}
\newcommand{\emp}{\tilde{p}}
\newcommand{\lr}{\alpha}
\newcommand{\reg}{\lambda}
\newcommand{\rect}{\mathrm{rectifier}}
\newcommand{\softmax}{\mathrm{softmax}}
\newcommand{\sigmoid}{\sigma} % Note: main text often hardcodes \sigma still, also
% many equations would likely frame bust if expanded to say \mathrm{sigmoid}
\newcommand{\sigm}{\sigmoid} % Duplicate command, this one is deprecated
\newcommand{\softplus}{\zeta} % Note: some figures may not use this macro, but main text consistently does
\newcommand{\KL}{D_{\mathrm{KL}}}
\newcommand{\Var}{\mathrm{Var}}
\newcommand{\standarderror}{\mathrm{SE}}
\newcommand{\Cov}{\mathrm{Cov}}
% Put these here so we can keep same syntax throughout the book but also change
% it easily.
% Wolfram Mathworld says $L^2$ is for function spaces and $\ell^2$ is for vectors
% But then they seem to use $L^2$ for vectors throughout the site, and so does
% wikipedia.
% No one really seems to use subscripts (l_2, L_2) on general math reference
% sites, though subscripts seem popular in machine learning.
\newcommand{\normlzero}{L^0}
\newcommand{\normlone}{L^1}
\newcommand{\normltwo}{L^2}
\newcommand{\normlp}{L^p}
\newcommand{\normmax}{L^\infty}

% NOTE: This one was not done with a macro originally, many uses need to converted
% to \parents rather than Pa if we want to change it.
\newcommand{\parents}{Pa} % See usage in notation.tex. Chosen to match Daphne's book.

\ifMIT
\else
\DeclareMathOperator*{\argmax}{arg\,max}
\DeclareMathOperator*{\argmin}{arg\,min}
\fi
\DeclareMathOperator{\sign}{sign}
\DeclareMathOperator{\Tr}{Tr}
\let\ab\allowbreak


\usepackage{enumitem}

\newcommand{\twovector}[2]{ \begin{smallmatrix} #1 \\ #2 \end{smallmatrix} }
\newcommand{\ptwovector}[2]{ \left( \twovector{#1}{#2} \right) }

\title{Exercises for Chapter 6: Deep Feedforward Networks}
\begin{document}
\maketitle

\section*{Which exercise goes with which section?}
\begin{itemize}
    \item Exercise \ref{DFN_ex_calculusofvariations}: 6.2.1.2 Learning Conditional Statistics.
    \item Exercise \ref{DFN_ex_bernoulli}: 6.2.2.2 Sigmoid Units for Bernoulli Output Distributions.
    \item Exercise \ref{DFN_ex_softmax}: 6.2.2.3 Softmax Units for Multinoulli Output Distributions.
    \item Exercise \ref{DFN_ex_linearhiddenunit}: 6.3.3 Other Hidden Units.
    \item Exercise \ref{DFN_ex_chainrule}: 6.5.2 Chain Rule of Calculus.
    \item Exercise \ref{DFN_ex_chainbackprop}: 6.5.3 Recursively Applying the Chain Rule to Obtain Backprop.
    \item Exercise \ref{DFN_ex_gradientsimplification}: 6.5.9 Differentiation Outside the Deep Learning Community.
    \item Exercise \ref{DFN_ex_Christianson}: 6.5.10 Higher-Order Derivatives.
\end{itemize}
    
\section*{Exercises}

%%%%%%%%%%%%%%%%%%%%%%%%%%%%%%
\begin{enumerate}
%%%%%%%%%%%%%%%%%%%%%%%%%%%%%%

%%%%%%%%%%%%%%%%%%%%%%%%%%%%%%
\item \label{DFN_ex_calculusofvariations} 
%%%%%%%%%%%%%%%%%%%%%%%%%%%%%%
This exercise investigates the two results from calculus of variations that are presented in this section. Consider the following finite data set of six data points where $x \in \R$ and $\vy \in \R^2$.

\begin{equation*}
\begin{array}{c|c||c|c}
    x & y & x & y \\
    \hline
    \hline
    1 & \ptwovector00 & 2 & \ptwovector00 \\
    \hline
    1 & \ptwovector12 & 2 & \ptwovector24 \\
    \hline
    1 & \ptwovector57 & 2 & \ptwovector{10}{14}
\end{array}
\end{equation*}

We will try to fit the data with a function $f(x) = \ptwovector{ax}{bx}$ for some $a, b \in \R$.

\begin{enumerate}
    \item Find $\E_{x,\vy\sim\pdata}\lVert \vy - f(x)\rVert^2$ in terms of $a$ and $b$.
    \item Find $f^* = \argmin_f \E_{x,\vy\sim\pdata}\lVert \vy - f(x)\rVert^2$.
    \item Show that $f^*$ computes the mean of $\vy$ for each value of $x$.
    \item Find $\E_{x,\vy\sim\pdata}\lVert \vy - f(x)\rVert_1$ in terms of $a$ and $b$.
    \item Find $f^* = \argmin_f \E_{x,\vy\sim\pdata}\lVert \vy - f(x)\rVert_1$.
    \item Show that $f^*$ computes the median of $\vy$ for each value of $x$.
\end{enumerate}

%%%%%%%%%%%%%%%%%%%%%%%%%%%%%%
\item \label{DFN_ex_bernoulli} 
%%%%%%%%%%%%%%%%%%%%%%%%%%%%%%
Let $y$ be a binary variable, so it only takes on values $0$ or $1$. Show that $\frac{\exp(yz)}{\sum_{y' = 0}^1 \exp(y'z)} = \sigma((2y-1))z$ in both cases ($y = 0$ or $y = 1$).  

%%%%%%%%%%%%%%%%%%%%%%%%%%%%%%
\item \label{DFN_ex_softmax} 
%%%%%%%%%%%%%%%%%%%%%%%%%%%%%%
This exercise develops intuition for the $\softmax$ function.
\begin{enumerate}
    \item Find $\softmax \ptwovector{7}{7}$.
    \item Find $\softmax \ptwovector{\ln 3}{0}$.
    \item Find $\softmax \ptwovector{9\ln 3}{\ln 3}$.
    \item Let $a$ be a real number. Find $\softmax \ptwovector{a + 8 \ln 3}{a}$.
    \item Use the mental shortcut that $\ln 20 \approx 3$ to find a quick estimate for $\softmax \left(\begin{smallmatrix} 6 \\ 3 \\ 0 \end{smallmatrix}\right)$.
\end{enumerate}

%%%%%%%%%%%%%%%%%%%%%%%%%%%%%%
\item \label{DFN_ex_linearhiddenunit} 
%%%%%%%%%%%%%%%%%%%%%%%%%%%%%%
Consider the linear map $\R^2 \to \R^3$ given by
\begin{equation*}
\begin{pmatrix} a \\ b \end{pmatrix} 
\mapsto      
\begin{pmatrix} a - b \\ 3(a - b) \\ 5(a - b) \end{pmatrix}.
\end{equation*}
\begin{enumerate}
    \item Write a matrix $\mW$ so that this map can be written as $\vy = \mW^T \vx$.
    \item Use the small rank of this map to find nice matrices $\mU$ and $\mV$ so that this map can be written as $\vy = \mV^T \mU^T \vx$.
    \item What are $p, q$ and $n$ here and why is it significant that $(n+p)q < np$?
\end{enumerate}

%%%%%%%%%%%%%%%%%%%%%%%%%%%%%%
\item \label{DFN_ex_chainrule} 
%%%%%%%%%%%%%%%%%%%%%%%%%%%%%%
Let $\vy = \left(\begin{smallmatrix} y_1\\y_2\\y_3 \end{smallmatrix}\right)$. We will define a function $z(\vy)$ and also a function $\vy(x)$, so that $z$ is in the end, a function of the real number $x$. Let
\begin{equation*}
    z\begin{pmatrix} y_1\\y_2\\y_3 \end{pmatrix} = 3y_1 + 4y_2 + 5y_3,
    \text{ and } 
    \vy(x) = \begin{pmatrix} x^2 \\ 3x \\ 8 \end{pmatrix}.
\end{equation*}
\begin{enumerate}
    \item Find an explicit representation of $z$ as a function only in terms of $x$.
    \item Find $\frac{dz}{dx}$ using your answer to the previous part.
    \item Find $\frac{dz}{dx}$ using the chain rule: $\frac{dz}{dx} = \sum_j \frac{\partial z}{\partial y_j} \frac{dy_j}{dx}$.
    \item Find $\frac{dz}{dx}$ by computing 
    $\left(\frac{\partial \vy}{\partial x}\right)^T \nabla_\vy z$.

\end{enumerate}

%%%%%%%%%%%%%%%%%%%%%%%%%%%%%%
\item \label{DFN_ex_chainbackprop} 
%%%%%%%%%%%%%%%%%%%%%%%%%%%%%%
Let $f(a) = a^2 + 1$ and consider the computational graph as in the text where $z = f(y), y = f(x), x = f(w)$. Our goal is to compute the derivative $\frac{dz}{dw}$ at $w = 1$ using two approaches.
\begin{enumerate} 
    \item Use $\frac{dz}{dw} = f'(y)f'(x)f'(w)$ to compute the derivative at $w=1$. Note that this requires more memory but less computation.
    \item Use $\frac{dz}{dw} = f'(f(f(w)))f'(f(w))f'(w)$ to compute the derivative at $w=1$. Note that this requires less memory but more computation.
\end{enumerate}

%%%%%%%%%%%%%%%%%%%%%%%%%%%%%%
\item \label{DFN_ex_gradientsimplification}
%%%%%%%%%%%%%%%%%%%%%%%%%%%%%%
Suppose we have variables $p_1, p_2, \ldots, p_n$ representing probabilities, and variables $z_1, z_2, \ldots, z_n$ representing unnormalized log probabilities. Suppose we define the softmax function 
\begin{equation*}
    q_j = \frac{\exp(z_j)}{\sum_k \exp(z_k)}
\end{equation*}
and construct a cross-entropy loss $J = -\sum_j p_j \log q_j$. 

\begin{enumerate}
    \item Show that $\frac{\partial}{\partial z_i} \log q_j = \phantom{1}-q_i$, when $i \neq j$.
    \item Show that $\frac{\partial}{\partial z_i} \log q_i = 1 - q_i$, where the indices match.
    \item Show that $\frac{\partial J}{\partial z_i} = q_i - p_i$, a simplification that a general back-propagation algorithm would miss.
\end{enumerate}

%%%%%%%%%%%%%%%%%%%%%%%%%%%%%%
\item \label{DFN_ex_Christianson}
%%%%%%%%%%%%%%%%%%%%%%%%%%%%%%
Let $f(\vx) = x_1^2 + x_1x_2$ and let $\vv = \ptwovector{a}{b}$. 
\begin{enumerate}
    \item Directly compute the Hessian matrix $\mH$ of $f$, and then find $\mH \vv$.
    \item Compute $\nabla_\vx\left[ \left(\nabla_\vx f(\vx) \right)^T \vv\right]$ as another way to find $\mH \vv$.
\end{enumerate}

%%%%%%%%%%%%%%%%%%%%%%%%%%%%%%
{\em Exercises contributed by Chris Cunningham}
\end{enumerate}
%%%%%%%%%%%%%%%%%%%%%%%%%%%%%%


\section*{Solutions}

%%%%%%%%%%%%%%%%%%%%%%%%%%%%%%
\begin{enumerate}
%%%%%%%%%%%%%%%%%%%%%%%%%%%%%%

%%%%%%%%%%%%%%%%%%%%%%%%%%%%%%
\item \label{DFN_ex_calculusofvariations_solution}
%%%%%%%%%%%%%%%%%%%%%%%%%%%%%%
\begin{enumerate}
    \item With only six data points, the expected value is $\frac16$ of the sum.
    \begin{align*}
        \E_{x,\vy\sim\pdata}\lVert \vy - f(x)\rVert^2 & = \frac16 \sum_{i=1}^6 \left(y^{(i)}_1-ax^{(i)}\right)^2 + \left((y^{(i)}_2-bx^{(i)}\right)^2 \\
        & = \frac{1}{6} \left(
                \begin{array}{c c}
                \twovector{\phantom{+}(0-a)^2}{+(0-b)^2} & \twovector{+(0 -2a)^2}{+(0 -2b)^2} \\
                \twovector{         + (1-a)^2}{+(2-b)^2} & \twovector{+(2 -2a)^2}{+(4 -2b)^2} \\
                \twovector{         + (5-a)^2}{+(7-b)^2} & \twovector{+(10-2a)^2}{+(14-2b)^2} \\
                \end{array}
                \right) \\
        & = \frac16 \left( 15a^2 - 60a + 15b^2 - 90b + 395 \right).
    \end{align*}
    \item The gradient with respect to $\vtheta = \ptwovector{a}{b}$ is
    \begin{align*}
        \nabla_\vtheta \E_{x,\vy\sim\pdata}\lVert \vy - f(x)\rVert^2 & = \frac16 \begin{pmatrix} 30a - 60 \\ 30b - 90 \end{pmatrix} \\
        & = 5 \begin{pmatrix}a - 2 \\ b - 3 \end{pmatrix}
    \end{align*}
    which is zero at $a = 2$, $b = 3$. Thus $f^*(x) = \ptwovector{2x}{3x}$.
    \item For $x = 1$ the mean value of $\vy$ is $\ptwovector{2}{3}$ and for $x = 2$ the mean value of $\vy$ is $\ptwovector{4}{6}$. Both satisfy $\vy = \ptwovector{2x}{3x}$, matching $f^*$ in the previous part.
    \item Again with six data points, the expected value is $\frac16$ of the sum.
    \begin{align*}
        \E_{x,\vy\sim\pdata}\lVert \vy - f(x)\rVert_1 & = \frac16 \sum_{i=1}^6 \lvert y^{(i)}_1-ax^{(i)}\rvert + \lvert y^{(i)}_2-bx^{(i)} \rvert \\
        & = \frac{1}{6} \left(
                \begin{array}{c c}
                \twovector{\phantom{+}\lvert 0-a \rvert}{+\lvert 0-b \rvert} & \twovector{+\lvert 0 -2a \rvert}{+\lvert 0 -2b \rvert} \\
                \twovector{         + \lvert 1-a \rvert}{+\lvert 2-b \rvert} & \twovector{+\lvert 2 -2a \rvert}{+\lvert 4 -2b \rvert} \\
                \twovector{         + \lvert 5-a \rvert}{+\lvert 7-b \rvert} & \twovector{+\lvert 10-2a \rvert}{+\lvert 14-2b \rvert} \\
                \end{array}
                \right) \\
        & = \frac36 \left( \lvert a \rvert + \lvert 1-a \rvert + \lvert 5-a \rvert + 
                           \lvert b \rvert + \lvert 2-b \rvert + \lvert 7-b \rvert \right) \\
        & = \frac12 \left( \lvert a \rvert + \lvert a-1 \rvert + \lvert a-5 \rvert + 
                           \lvert b \rvert + \lvert b-2 \rvert + \lvert b-7 \rvert \right)
        \end{align*} 
    \item Let $N$ be the expected value computed in the previous question. Note that for $a < 1, \frac{\partial N}{\partial a} < 0$ and for $a > 1, \frac{\partial N}{\partial a} > 0$. Similarly for $b < 2, \frac{\partial N}{\partial b} < 0$ and for $b > 2, \frac{\partial N}{\partial b} > 0$. Thus the minimum is at $a = 1, b = 2$. 
    
    Further note that $N$ is not differentiable, so the gradient is not guaranteed to find its minimum. However, it happens to work here: the gradient with respect to $\vtheta = \ptwovector{a}{b}$ is
    \begin{equation*}
        \nabla_\vtheta \E_{x,\vy\sim\pdata}\lVert \vy - f(x)\rVert_1 = \frac12 \begin{pmatrix} \sign(a) + \sign(a-1) + \sign(a-5) \\
                                                 \sign(b) + \sign(b-2) + \sign(b-7) \end{pmatrix}
    \end{equation*}
    which is zero exactly at $a = 1, b = 2$. 
    \item For $x = 1$ the median value of $\vy$ is $\ptwovector{1}{2}$ and for $x = 2$ the mean value of $\vy$ is $\ptwovector{2}{4}$. Both satisfy $\vy = \ptwovector{1x}{2x}$, matching $a=1, b=2$ in the previous part.
\end{enumerate}

%%%%%%%%%%%%%%%%%%%%%%%%%%%%%%
\item \label{DFN_ex_bernoulli_solution} 
%%%%%%%%%%%%%%%%%%%%%%%%%%%%%%
Regardless of $y$, 
\begin{align*}
        \frac{\exp(yz)}{\sum_{y' = 0}^1 \exp(y'z)} 
    & = \frac{\exp(yz)}{\exp(0z) + \exp(1z)}  \\
    & = \frac{\exp(yz)}{1        + \exp(z)}.
\end{align*}

If $y = 0$, then this is $\frac{1}      {1 + \exp(z)} = \sigma(-z)$.

If $y = 1$, then this is $\frac{\exp(z)}{1 + \exp(z)}$. Divide by $\exp(z)$ to get $\frac{1}{\exp(-z) + 1} = \sigma(z)$.

This is exactly the function 
\begin{equation*}
    \sigma((2y-1)z) = \begin{cases} \sigma(-z) & \text{ if } y = 0 \\ 
                                    \sigma(z)  & \text{ if } y = 1.
                      \end{cases}
\end{equation*}

%%%%%%%%%%%%%%%%%%%%%%%%%%%%%%
\item \label{DFN_ex_softmax_solution} 
%%%%%%%%%%%%%%%%%%%%%%%%%%%%%%
\begin{enumerate}
    \item $\softmax \ptwovector{7}{7}$ has the same thing in both arguments: $\frac{e^7}{e^7 + e^7} = \frac12$. So $\softmax \ptwovector{7}{7} = \ptwovector{0.5}{0.5}$.
    \item In the first slot we have $\frac{\exp(\ln 3)}{\exp(\ln 3) + \exp(0)} = \frac{3}{4}$. 
    
    In the second slot we have $\frac{\exp(0)}{\exp(\ln 3) + \exp(0)} = \frac{1}{4}$.
    
    Thus $\softmax\ptwovector{\ln 3}{0} = \ptwovector{0.75}{0.25}$.
    
    \item In the first slot we have $\frac{\exp(9 \ln 3)}{\exp(9 \ln 3) + \exp(\ln 3)} = \frac{3^9}{3^9 + 3} = \frac{3^8}{3^8+1}$.
    
    In the second slot we have $\frac{\exp(\ln 3)}{\exp(9 \ln 3) + \exp(\ln 3)} = \frac{3}{3^9 + 3} = \frac{1}{3^8+1}$.
    
    Thus $\softmax \ptwovector{9\ln 3}{\ln 3} \approx \ptwovector{0.9998}{0.0002}$.
    
    \item This is the same input as the previous part, but with both inputs shifted by $a - \ln 3$, so it has the same output: $\softmax \ptwovector{a + 8\ln 3}{a} \approx \ptwovector{0.9998}{0.0002}$. To see the cancellation in the first argument worked out:
    
    \begin{equation*}
    \frac{\exp(a + 8\ln 3)}{\exp(a + 8\ln 3)+\exp(a)} = \frac{\exp(a) \cdot 3^8}{\exp(a) \cdot 3^8 + \exp(a)} = \frac{3^8}{3^8 + 1}.
    \end{equation*}
    
    \item \begin{align*}
              \softmax \left(\begin{smallmatrix} 
                          6               \\ 3              \\ 0 
                       \end{smallmatrix} \right) 
    & \approx \softmax \left(\begin{smallmatrix} 
                          2 \ln 20        \\ \ln 20         \\ 0 
                       \end{smallmatrix} \right) \\
    & =       \softmax \left(\begin{smallmatrix} 
                          \ln 400         \\ \ln 20         \\ \ln 1 
                       \end{smallmatrix} \right) \\
    & =                \left(\begin{smallmatrix} 
                          \frac{400}{421} \\ \frac{20}{421} \\ \frac{1}{421} 
                       \end{smallmatrix} \right) \\
    & \approx          \left(\begin{smallmatrix} 
                          \frac{19}{20}   \\ \frac{1}{20}   \\ \frac{1}{400} 
                       \end{smallmatrix} \right) \\
    & =                \left(\begin{smallmatrix} 
                          0.9500          \\ 0.0500         \\ 0.0025 
                       \end{smallmatrix} \right).
    \end{align*}
                     
    Of course the true values add up to 1; if we calculate with no approximations until the end, we can see that the quick estimate above is quite close:
    
    $       \softmax \left(\begin{smallmatrix} 
                        6        \\ 3      \\ 0 
                     \end{smallmatrix} \right)
    =                \left(\begin{smallmatrix} 
                        \frac{e^6}{e^6+e^3+e^0} & 
                        \frac{e^3}{e^6+e^3+e^0} & 
                        \frac{e^0}{e^6+e^3+e^0} 
                     \end{smallmatrix}\right)^T
     \approx         \left(\begin{smallmatrix} 
                        0.9503    \\ 0.0473 \\ 0.0024 
                     \end{smallmatrix} \right)$.              
\end{enumerate}

%%%%%%%%%%%%%%%%%%%%%%%%%%%%%%
\item \label{DFN_ex_linearhiddenunit_solution} 
%%%%%%%%%%%%%%%%%%%%%%%%%%%%%%
\begin{enumerate}
    \item $\mW = \begin{pmatrix} \phantom{-}1 & \phantom{-}3 & \phantom{-}5 \\ -1 & -3 & -5 \end{pmatrix}$.
    \item The clue is that the transformation is rank 1, so we should use $q = 1$. This leads us to 
    
    $\mU = \begin{pmatrix} \phantom{-}1 \\ -1 \end{pmatrix}, 
           \mV = \begin{pmatrix} 1 & 3 & 5 \end{pmatrix}$.
    \item The model has two inputs, so $n = 2$. The model has three outputs, so $p = 3$. The entire model has rank 1 due to the action of $\mU$ and $\mV$, so $q = 1$. 
    
    When training a model to match this one, using $\vy = \mW^T \vx$ requires the model to learn $np = 6$ parameters, but when using a linear hidden unit so that $\vy = \mV^T \mU^T \vx$, the model only has to learn $(n+p)q = 5$ parameters. This effect would be even more pronounced for larger values of $n$ and $p$.
\end{enumerate}

%%%%%%%%%%%%%%%%%%%%%%%%%%%%%%
\item \label{DFN_ex_chainrule_solution} 
%%%%%%%%%%%%%%%%%%%%%%%%%%%%%%
\begin{enumerate}
    \item $z(x) = 3x^2 + 12x + 40$.
    \item $\frac{dz}{dx} = 6x + 12$.
    \item $\frac{dz}{dx} = 3\cdot2x + 4\cdot 3 + 5\cdot 0 = 6x + 12$.
    \item $\frac{dz}{dx} = \begin{pmatrix} 3 & 4 & 5 \end{pmatrix} \cdot \begin{pmatrix} 2x \\ 3 \\ 0 \end{pmatrix} = 6x + 12$.
\end{enumerate}

%%%%%%%%%%%%%%%%%%%%%%%%%%%%%%
\item \label{DFN_ex_chainbackprop_solution} 
%%%%%%%%%%%%%%%%%%%%%%%%%%%%%%
\begin{enumerate}
    \item Since $f(a) = a^2 + 1, f'(a) = 2a$. We can then find all the values of the variables:
    \begin{align*}
        w & = 1 \\
        x & = w^2 + 1 = 2 \\
        y & = x^2 + 1 = 5
    \end{align*}
    then find the derivatives in question:
    \begin{align*}
        f'(w) & = 2w = 2 \\
        f'(x) & = 2x = 4 \\
        f'(y) & = 2y = 10
    \end{align*}
    Thus $\frac{dz}{dw} = 2\cdot4\cdot10 = 80$. This process used a significant amount of memory due to its intermediate computations, but all the intermediate computations were simple operations.
    \item This time we avoid intermediate computations:\begin{align*}
    \frac{dz}{dw} & = f'(f(f(w)))                f'(f(w))            f'(w) \\
                  & = 2\left((w^2+1)^2+1 \right) 2\left(w^2+1\right) 2(w) \\
                  & = 2\left(w^4+2w^2+1+1\right) 2\left(w^2+1\right) 2(w) \\
                  & = 8\left(w^4+2w^2+2  \right)  \left(w^2+1\right)  (w) \\
    \left.\frac{dz}{dw}\right\vert_{w=1} 
                  & = 8\left(5\right)\left(2\right)(1) \\
                  & = 80.
    \end{align*}
    This process used less memory (we stored no intermediate values), but we unnecessarily computed $f(w) = w^2+1$ multiple times.
    
\end{enumerate}

%%%%%%%%%%%%%%%%%%%%%%%%%%%%%%
\item \label{DFN_gradientsimplification_solution}
%%%%%%%%%%%%%%%%%%%%%%%%%%%%%%
\begin{enumerate}
    \item We know $q_j = \frac{\exp(z_j)}{\sum_k \exp(z_k)}$, so we can simplify $\log q_j = z_j - \log \sum_k \exp(z_k)$. Then we can take derivatives:
    \begin{align*}
        \frac{\partial}{\partial z_i} \log q_j 
        & = \underset{(i\neq j)}{0}           - \frac{1}{\sum_k \exp(z_k)} \frac{\partial}{\partial z_i} \sum_k \exp(z_k) \\
        & = \phantom{\underset{(i\neq j)}{0}} - \frac{1}{\sum_k \exp(z_k)} \exp(z_i) \\
        & = - q_i.
    \end{align*}
    \item Again simplify $\log q_i = z_i - \log \sum_k \exp(z_k)$, then take derivatives:
    \begin{align*}
        \frac{\partial}{\partial z_i} \log q_i 
        & = \underset{(i= j)}{1} - \frac{1}{\sum_k \exp(z_k)} \frac{\partial}{\partial z_i} \sum_k \exp(z_k) \\
        & = 1 - \frac{1}{\sum_k \exp(z_k)} \exp(z_i) \\
        & = 1 - q_i.
    \end{align*}
    \item Since $J = - \sum_j p_j \log q_j$, the partial derivatives of this sum are linear combinations of the answers to the previous parts.
    \begin{align*}
        \frac{\partial J}{\partial z_i} 
        & = +p_1q_i + p_2q_i - \cdots -p_i(1-q_i) + \cdots \\
        & = -p_i + \sum_j p_jq_i \\
        & = -p_i + \left(\sum_j p_j\right) q_i, \text{ but } \sum_j p_j = 1, \text{ so} \\
        & = q_i - p_i.   
    \end{align*}
\end{enumerate}

%%%%%%%%%%%%%%%%%%%%%%%%%%%%%%
\item \label{DFN_ex_Christianson_solution}
%%%%%%%%%%%%%%%%%%%%%%%%%%%%%%
\begin{enumerate}
    \item $\mH = \begin{pmatrix} 2 & 1 \\ 1 & 0 \end{pmatrix}$, so $\mH \vv = \begin{pmatrix} 2a+b \\ a \end{pmatrix}$.
    \item If it were significantly easier to compute $\nabla_\vx$ than $\mH$, you might prefer this path:\begin{align*}
        \phantom{\nabla_\vx[(}\nabla_\vx f(\vx)\phantom{)^T \vv]} & = \ptwovector{2x_1 + x_2}{x_1} \\
        \phantom{\nabla_\vx[}(\nabla_\vx f(\vx))^T \vv\phantom{]} & = 2x_1a + x_2a + x_1b \\
                 \nabla_\vx[(\nabla_\vx f(\vx))^T \vv]            & = \ptwovector{2a+b}{a}.
    \end{align*}
\end{enumerate}

%%%%%%%%%%%%%%%%%%%%%%%%%%%%%%
{\em Solutions contributed by Chris Cunningham}
%%%%%%%%%%%%%%%%%%%%%%%%%%%%%%

%%%%%%%%%%%%%%%%%%%%%%%%%%%%%%
\end{enumerate}
%%%%%%%%%%%%%%%%%%%%%%%%%%%%%%

\end{document}