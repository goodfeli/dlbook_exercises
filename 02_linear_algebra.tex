%!TEX root = all_exercises.tex
\beforeFirstExercises{ch2}

exercises (if any) for each section would go here... see sample in ch3

\afterLastExercises{ch2}


\setcounter{section}{12}
\section{Linear Algebra Problem}

\begin{problems}{ch2}


	\begin{problem}
		The expression $\alpha \vu$ for $\alpha \in \R$ and
		unit vector $\vu \in \R^n$ defines a line of points that may be obtained by varying the
		value of $\alpha$.
		Derive an expression for the point $\vy$ that lies on this line that is
		as close as possible to an arbitrary point $\vx \in \R^n$.
		This operation of replacing a point by its nearest member within some set
		is called {\em projection}.
		
		{\em Exercise contributed by Ian Goodfellow}
		
		\begin{solution}
			We begin by defining the distance from $\vy$ to $\vx$.
			We would like to find the $\vy$ that minimizes this distance:
			\begin{equation}
			|| \vx - \vy||^2.
			\end{equation}
			Next, we need to enforce the constraint that $\vy$ lies on the line defined by $\alpha \vu$.
			We can do this simply by defining $\vy$ to be $\alpha \vu$.
			\begin{equation}
			|| \vx - \alpha \vu ||^2.
			\end{equation}
			Next, we expand the expression:
			\begin{align}
			& || \vx - \alpha \vu ||^2 \\
			=& ( \vx - \alpha \vu )^\top (\vx - \alpha \vu) \\
			=& \vx^\top \vx - 2 \alpha \vx^\top \vu + \alpha^2 \vu^\top \vu \\
			=& \vx^\top \vx - 2 \alpha \vx^\top \vu + \alpha^2.
			\end{align}
			In the last line,
			we used the fact that $\vu$ is a unit vector to make the simplification $\vu^\top \vu = 1$.
			
			We can minimize this distance by taking the derivative with respect to $\alpha$ and setting it to zero:
			\begin{align}
			& - 2  \vx^\top \vu + 2 \alpha = 0 \\
			\Rightarrow & \alpha = \vx\top \vu.
			\end{align}
			
			Recalling that $\vy = \alpha \vu$, we can conclude that $\vy = \vx^\top \vu \vu$.
			
			{\em Solution contributed by Ian Goodfellow}
		\end{solution}
	\end{problem}


	\begin{problem}

		\begin{hint}
		\end{hint}

		\begin{answer}\end{answer}

		\begin{solution}
		\end{solution}
	\end{problem}



	\begin{problem}

		\begin{hint}
		\end{hint}

		\begin{answer}\end{answer}

		\begin{solution}
		\end{solution}
	\end{problem}



	
\end{problems}


